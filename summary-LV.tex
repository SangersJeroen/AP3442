% ----------------------- TODO ---------------------------
% Change per hand-in
\newcommand{\NUMBER}{1} % exercise set number
\newcommand{\EXERCISES}{5} % number of exercises

\newcommand{\COURSECODE}{AP3442}
\newcommand{\TITLE}{Questions \& Answers 2022, LV}
\newcommand{\STUDENTA}{Jeroen Sangers - Applied Physics}
\newcommand{\DEADLINE}{June 26th, 2022}
\newcommand{\COURSE}{Quantum Hardware II}
% ----------------------- TODO ---------------------------

\documentclass[a4paper]{scrartcl}

\usepackage[utf8]{inputenc}
\usepackage[british]{babel}
\usepackage{amsmath}
\usepackage{amssymb}
\usepackage{fancyhdr}
\usepackage{color}
\usepackage{graphicx}
\usepackage{lastpage}
\usepackage{listings}
\usepackage{tikz}
\usepackage{pdflscape}
\usepackage{subfigure}
\usepackage{float}
\usepackage{polynom}
\usepackage{hyperref}
\usepackage{tabularx}
\usepackage{forloop}
\usepackage{geometry}
\usepackage{listings}
\usepackage{fancybox}
\usepackage{enumitem}
\usepackage{tikz}
\usepackage{algpseudocode,algorithm,algorithmicx}
\usepackage{fontspec}


\setmainfont{Baskerville Light}[
	BoldFont	= Baskerville Bold ,
	ItalicFont	= Baskerville Light-Italic
]
% Algorithm command
\newcommand*\Let[2]{\State #1 $\gets$ #2}

% Matrix notation
\newcommand{\matr}[1]{\mathbf{#1}}

% Margins
\geometry{a4paper,left=3cm, right=3cm, top=3cm, bottom=3cm}

% Header and footer setup
\pagestyle {fancy}
%\fancyhead[L]{Tutor: \TUTOR}
\fancyhead[L]{\TITLE}
\fancyhead[C]{\STUDENTA}
\fancyhead[R]{\today}

\fancyfoot[L]{\COURSECODE}
\fancyfoot[C]{\COURSE}
\fancyfoot[R]{Page \thepage /\pageref*{LastPage}}

% Formatting of "title"
\def\header#1#2{
  \begin{center}
    {\Large Exercise set}\\
    {(Deadline #2)}
  \end{center}
}

\newcommand{\qa}[2]{#1\\ \textit{#2}}

\begin{document}
\subsection*{Quantum dots}
\begin{enumerate}[label=(\alph*)]
  \item \begin{enumerate}[label=(\roman*)]
    \item \qa{How can individual electrons be trapped in semiconductor quantum dots?}{ans}
    \item \qa{What are some common material platforms?}{ans}
  \end{enumerate}

  \item \begin{enumerate}[label=(\roman*)]
    \item \qa{What does the charge stability diagram of a double quantum dot look like?}{ans.}
    \item \qa{What physical effects do we see?}{ans.}
    \item \qa{How can we measure a charge stability diagram?}{ans.}
  \end{enumerate}

  \item \qa{Explain how the state of a single electron spin in a quantum dot can be read out. What experimental conditions need to be met for the read-out to achieve high fidelity?}{}
  \item \qa{Explain two methods for the coherent control of a single electron spin in a quantum dot. Discuss the advantages and disadvantages of both. Bonus: explain a third method.}{}
  \item \qa{Explain two types of two-qubit gates between single-electron spin qubits in quantum dots. Discuss the advantages and disadvantages of both. Bonus: explain a third method.}{}
  \item \begin{enumerate}[label=(\roman*)]
    \item \qa{What are the limiting decoherence mechanism for single-spin qubits in quantum dots?}{}
    \item \qa{How do they impact the fidelity of single-shot readout and of single- and two-qubit gates?}{}
    \item \qa{How did the timescales and limiting mechanisms evolve over the years?}{}
    \item \qa{To what extent can dynamical decoupling techniques extend the decoherence times?}{}
    \item \qa{What does this tell us about the decoherence mechanisms?}{}
  \end{enumerate}
\end{enumerate}
\subsection*{Nitrogen vacancy qubits}
\subsection*{Superconducting qubits}
\subsection*{Ion trap qubits}
\subsection*{Adiabatic quantum computing}

\end{document}